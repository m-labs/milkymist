\documentclass[a4paper,11pt]{article}
\usepackage{fullpage}
\usepackage[latin1]{inputenc}
\usepackage[T1]{fontenc}
\usepackage[normalem]{ulem}
\usepackage[english]{babel}
\usepackage{listings,babel}
\lstset{breaklines=true,basicstyle=\ttfamily}
\usepackage{graphicx}
\usepackage{moreverb}
\usepackage{url}

\title{WISHBONE Block RAM}
\author{S\'ebastien Bourdeauducq}
\date{\today}
\begin{document}
\maketitle{}
\section{Specifications}
This core creates 32-bit storage RAM on the WISHBONE bus by using FPGA Block RAM. Its contents can be initialized and byte-wide writes are supported. Burst access is not supported. The typical use is to provide boot ROM for softcore CPUs.

\section{Using the core}
You should specify the block RAM storage depth, in bytes, by using the \verb!adr_width! parameter.

For initialization, four files must be provided, using the \verb!mem0_file_name!, \verb!mem1_file_name!, \verb!mem2_file_name! and \verb!mem3_file_name! parameters. When these files are read, the word \verb!x! in the 32-bit memory is initalized with the value formed by the 8-bit ASCII hexadecimal values read at line \verb!x! in each file, with \verb!mem3_file_name! providing the most significant byte and \verb!mem0_file_name! the least significant one.

If these parameters are set to the particular value \verb!none!, then the memory is not initialized.

\end{document}
